\chapter{Spreadsheets as serious computation}
People may see spreadsheets early, but they are liable to be presented as
tools for simple summing of columns, or perhaps for sorting rows. I.e. as
just slightly animated mete tables of numbers that are not too
interesting. They may also be seen as natural tools for a statistician
who needs to reduce a pile of numbers to something. But through the eyes
of a Computer Scientist they provide more opportunities!


\begin{enumerate}
\item Given a tabulation of an unknown function, taking successive differences lets you detect if it is in fact a polynomial and find its degree, and you can the very easily extrapolate it.
\item Given the tabulation of a function, put a small error into one value in the column. How can you find it? Take differences again and after a while it will be very visible!
\item Given a slowly converging sequence you may fear it will take a lot of work to evaluate enough terms that you can find its limit. But I have seen Geoff Miller use simple convergence accelerating techniques to compute ? to around 10 significant figures doing all the working by hand on the blackboard. I was impressed then! Spreadsheets are needed now for those whose mental arithmetic and accuracy do not match those displayed by JCPM.
\item The topics covered in Iterated Functions provide further possibilities here. 
\end{enumerate}

