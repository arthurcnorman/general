\chapter{Ingenious Data Structures}
\begin{quotation}\textit{
If challenges to set up a computer system to perform some action or
data transformation it can be natural to stary by asking ``How would
I do this by hand?''. Often that woudl let you write a simple and direct
program to do the job. However there are many many cases where more
alaborate procedures will do a much better job. Especially when you then
ask just how far can these improvements go it becomes possible to
find amazingly ingenious and complicated ways to solve the original
problem. Sometimes these even give the best practical results!
}\end{quotation}
Textbook of algorithms are often full of explanations of clever ways
to do things that would not at first have been at all obvious. These
have typically been invented because the more straightforward ways
of solving the problems concerned can be improved on, often by
huge amounts once you get to big enough test cases. So in this
chapter a few of these schemes are presented as an ilustration of the
way in which seeking the most efficient scheme can escalate difficulty quit
a lot.

\section{Binomial Heaps}
A problem that can have plenty of real-world application is the
maintainance of a priority queue. With such a queue anybody
joining the queue has an associated weight or priority. The
queue is processed by insisting that whenever the server finishes
with one customer they next look after whichever one in the queue has
highest priority. Clearly this scheme arranges that a new customer but
very important customer can arrive and jump much of the queue. There are
two main operations whose cost needs to be considered: inserting
a new customer into the queue and identifying and removing the next
to be served.

The most naive scheme will be to keep all customers in the queue
in a list arranged such that adding a new one has unit cost -- and not
at that stage worrying about the priorities. Then to pick the next
one to serve it may be necessary to scan the whole of that list to
identify the most worthy member of it, and excise them from the
list regardless of whether they are its is head, tail or somewhere in the
middle. That may have optimised adding new customers to the queue
but it makes selecting the next one to be served have a cost proportionate
to the queue length. If this was in fact the best that coule be achieved
all would be simple, but a datastructure called a ``heap'' improves on
it sharply making the cost of both queue activities proportional to
the logarithm of the queue length. For long queues this gives a very
good saving.

The explanation of heaps given here is not going to go into all the
details and tricks that proper textbooks do because the main payoff
of this section is something that builds on the general idea. So for
now a heap is a structure arranged as a (binary) tree. Each node
of the tree holds a customer and references to two sub-trees. Two key
rules ar applied: the customer in each node is one who will have priority
over every customer in either of the two sub-trees. And things are
arranged so that the number of customers in each sub-tree are
close to the same.  The first of these means that server has
immediate access to the most important customer -- they are the one
in the top node of the tree. The second ensures that the tree
is nicely balanced and that if there are $N$ customers in all the
the height of the tree is only $\log_{2}(N)$\footnote{Well that logarithm
usually has a value that is not a whole number, so we need to round
it up to get one!}. The management of such a heap is based on
needing to be able to add new items to it while preserving its
properties in time proportionate to its height, and equally
being able to repair it when its top item is removed in a simmilarly
fast way. Go and read the books to discover how that can be achieved,
because the interest here is in making the problem slightly harder
yet in a way that calls for further ingenuity.

Imagine you now have priority queues all sorted, and your setup now
happens to have two servers each with their own separate queue. You
are not allowed to make any assumption about which arrriving customers
joined which queue. Now one server finishes their shift, and the
remaining one is left to handle all the work. This immediatly calls
for the two separate priority queues to be consolidated. One could
do that by asking each customer who has been abandoned by the server
they were waiting for to join the other queue one at at time, but
since the queues we are now thinking about represent the queues as
trees it is likely that this has a significant cost. The scheme
sketched next reduces that almost as much as is possible!

Priority queues that may need to be merged can usefully be represented
by a data structure known as a ``Binomial Heap''. All operations on
such heaps have costs no worse that the logarithm of the number of
items stored. The explanation here will start from the top level
so that it can motivate the data structure used by showing why it is
good.

The key clever idea here is that if we have any number $N$ we can look
at how it would be written in binary notation and express it as the
sum of a bunch of powers of 2. So for instance $39$ is $100111$ in
binary and that means $39 = 2^0 + 2^2 + 2^2 + 2^5$. If the number is $N$
then we can saay that there are $\log N$ bits in its binary representation.

If a Binomial Heap (which is going to act as a priority queue) has
$N$ items stored in it then they are arranged in a buunch of
sub-heaps each of which has size that is a power of 2. It is not yet
clear just how these will be represented, but it will be arranged that the
higest priority item in each sub-heap is instantly accessible. That
means that the top item in the whole heap can be found by checking each
of the (up to) $\log N$ sub-heaps.

The clever part now emerges when you wish to consolidate two such
heaps into one. The steps taken follow exactly the pattern used in
performing addition on binary values. It can consider each possible
power of 2 in turn. If neither input has a sub-heap that size
then the output will not. If just one has then that will appear
in the result. But if both do then those two sub-heaps get
consolidated in a way that will be descrribed soon into a single
one that corresponds to the next power of 2 up. In terms of
the binary addition this is a ``carry''. Provided that the sub-heaps are
kept with the $2^0$ one first and provided consolidating a pair
of sub-heaps on size $2^j$ into a single one of size $2^{j+1}$ is
cheap this manages to add (or pergaps we say form the union) of the
two binomial heaps in logarithmic time.

Now what about that consolidation step? Well a good way to represent
a sub-heap of size $2^j$ is to have the highest priority item in it
picked out and sitting at the top, and the remaining $2^j - 1$ items
kept in a list of smaller heaps of size $2^{j-1}, 2^{j-2}, \ldots 4, 2, 1$.
Happily this satisfies our hope that the top item in the sub-heap would
be easy to find, and it means that the double size sub-tree can be
formed very easily by comparing the top items in the two
trees to me merged and just pushing the smaller one onto the
list held by the larger.

The task of removing the top item from a heap is equally straightforward.
We alreadyu know we can identify which sub-heap had the desired
element at its head. Remove that whole sub-heap from the top-level
list. Now if you lop the top item from the bit of structure you have
just retrieved you have a nice list of sub-heaps each of whose size
is a power of two. Gosh that is just the shape of a geberal Binomial
Heap and you can re-insert all its data into the main one using
just the binary addition process already described.

Those who are properly pedantic will observe that in each of the
various sub-heaps you will wany to have stored not just the top
element and not just a reference to the list of sub-sub heaps, but
something to explain how many items are present (i.e.\ which power of
2 is involved) and probably also a reference to the tail end of the
chain of sub-sub-heaps so that tagging a new item on the end is
really cheap. Doing all that carries some overhead but for cases with
enough customers the savings by having costs that grow only
logarithmically with the size of the queues is so much more valuable
that it is not a big issue.

The tricks and the elaboration of data representation here may seem
extreme enough that it would be natural to expect it was the best that
could be achieved. But massochists can look in the bext chapter
of their Big Book of Algorithms to learn about ``Fibonacci Heaps'' that
are yet more bizarre -- and which may only rather rarely be useful
in practise because although from a theoretical analysis its costs
grow slowly in pracise the overheads mean that simpler schemes test
to win. But for anybody keen to see how difficult computing
can be made looking at the them, and at the Brodal Queue and all
other options for priority queue implementation can provide a fine
collection of rabbit holes to dive into.
\section{Continued Fractions}
Any rational number $P/Q$ will have a decimal representation that
either terminates or that after some starting digits continues into
a repeating pattern. In contrast numbers such as $e$ and $\pi$ lead
to non-repeating patterns of digits. There is another way of expressing
numbers -- continued fractions. This has the property that rational
numbers always have a terminating representation, but values that are
roots of quadratic equations (that have whole number coefficients) end
up with repeating patterns.
section{Bit counts}
Consider the three sequences
\begin{verbatim}
 (1)   1 2 4 8 16 32 64 128 256 ...
 (2)   3 5 6 9 10 12 17 18 20 24 ...
 (3)   7 11 13 14 19 21 22 25 26 ...
\end{verbatim}
where the first sequence is easy to recognize but the next two may
at first be less obvious. However once there are displayed in binary
things start to become clear:
\begin{verbatim}
 (1)   1 10 100 1000 10000 100000 1000000 ...
 (2)   11 101 110 1001 1010 1100 10001 10010 ...
 (3)   111 1011 1101 1110 10011 10101 10110 ...
\end{verbatim}
and it is now easy to see that each sequence collects the numbers whose
binary representations have a certain number of 1 bits. Suppose that
following one of these sequences (including the follow-on ones that cover
4, 5 and so on bits) can allow one to enumerate all the ways that $r$
things can be selected from a set of size $n$. Often in the mathematics
of combinatorics the greatest concern will be how many ways of
picking $r$ from $n$ there will be and looking at just what all the
cases are is not terribly important -- just jowning that a binomial
coefficient lets us count is enough. But with computers it is entirely
plausible that it will sometimes be useful to work through all the
individual cases. This generating the above sequences and their friends
might be useful.

Implementing code that will step through the values at first looks
a bit messy, but amazingly if one mixes up oridinary arithmetic
operations with \verb@&@, \verb@|@ and \verb@^@ for bitwise and, or
amd exclusive or it can be done an an amazingly concise way,
as reported in Hakmem\cite{hakmem} and the Hacher's
Delight\cite{hackersdelight}.
\begin{verbatim}
    [start off with a number in n]
    w1 := n & -n;
    w2 := n + w1;
    n := (((w2 ^ n) >> 2) / w1) | w2;
    [n is now the next value up with the same bit-count]
\end{verbatim}
It is also sometimes useful to note that the division (which might
be a bit slower than other basic arithmetic on some conputers) is
dividing by a power of 2, and so when the computer provides a way
to discover the position on the single bit that will be present in
\verb@w1@ it will be possible to replace the division by a right
shift (\verb@>>@) of the bits involved. So a first challenge set here is
to understand exactly why the 3-line program fragment there works! But then
the real issue is going be to sort out just what else can be achieved
if arithmetic and logical operations on machine binary numbers are
combined. The Hacker's Delight has a range of examples where at least
if remainder calculations are allowed demonstrate how counting the
bits in a word, reversing them and other things are possible.
\section{What next?}
An INCOMPLETE section...
\begin{quotation}\textit{
The idea here is to go all the wey back to HAKMEM and report on
performing arithmetic on continued fractions. Along the way II want to
present the fact that many values depending on $e$ have continued fractions
that interleave an arithmetic progression with other simple patterns.
Eg $e$ has partial quotients 2,1,2,1,1,4,1,1,6,1,1,8,1,1,8,...
}\end{quotation}

