\chapter{PIC microcontrollers}
The 18F2550 PIC microcontroller costs around a couple of quid on ebay and
presumably at educational prices in bulk they are ``cheap as chips;;. If we
could stick one on a board with a USB socket, 2 push-buttons, 2 leds and
a header of some sort so that most pins were just available for off board
use (plus a 4-pin connector for use with a dedicated programmer) we could
put a USB bootloader in and what you then have is something much like the
US\$25 commercial offering  you can see at
\verb+http://www.sparkfun.com/commerce/product_info.php?products_id=8265+. If
this could be offered to participants for 10 quid (which almost sounds
feasible) and we hand a bunch here it could form the basis for in
introduction to a wide range of jolly things. The pragmatics here may be
harder than for the earlier examples I have given but the scope opened up
is huge.

\begin{enumerate}
\item Machine code programming of a PIC, with all the good games one can
play there
\item interfacing the host computer to other devices, both analog and digital
\end{enumerate}

