\chapter{Regular Languages}
This is obviously related to the previous topic, but is here because it
is an area where there are few prerequisites, where results are maybe fun
and where puzzles abound. It starts with an explanation of what regular
expressions are (so easy!) and probably talks through the Pumping Lemma
(that many will find it hard to absorb all at once) as the ``gee-whiz,
Computer Science has some real clever stuff in it'' bit.

\begin{enumerate}
\item The puzzles/problems set can be styled after {\em here is a language
 described in words --- can you find a regular expression for it or can you
convince yourself that there is not one}.
\item Here is a two regular expressions. Can you produce regular
expressions that are for the complement of each and the the intersection
of the two languages?
\item Here is a regular expression, are there any strings that it does not
 match?
\item Here are two machines, or two regular expressions, or one of each.
Do they accept the same language?
\item Especially if you permit intersection and complementation as extra
constructors, try to find a regular expression written in $n$ symbols where
that shortest non-empty string it can accept is as long as possible.
Eg \verb+(aaa*)&(aaaaa)*+ matches strings of lengths that are both
multiplies of 3 and of 5, hence only multiples of 15. Can you do better
[the answer is yes, if your RE is of length $k$ you can make it match
strings of length over $2^{2^{2^{2^{...2^{k}}}}}$ for any height tower of exponents
you like, so this seemingly harmless problem is not quite elementary!]
\end{enumerate}
