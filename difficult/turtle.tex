\chapter{Turtle Graphics}

One of the schemes often proposed for getting the very young into
computation involved letting them draw pictures using a ``turtle''.
This moves around the world leaving a trail that shows where it has been
(so why is it not described as a snail?). It is possible to instruct it
to move directly forward by some number of steps or to turn left or
right by an angle that is usually specified in degrees. Combinations
of these two operations can be repeated. So two very easy initial examples
of what can be done are
\begin{verbatim}
  repeat 4 times
    move forward by 10
    turn left by 90 degrees
\end{verbatim}
end
\begin{verbatim}
  repeat 5 times
    move forward by 10
    turn left by 90
    move forward by 10
    turn right by 90
\end{verbatim}
where one of these draws a square and the other a zig-zag. By giving
instructions that are less repetitive it will be possible to draw a house
or other interesting outlines. It can be useful to be able to say
``pen up'' and ``pen down'' so that the drawing being produced does not
have to use a single continuous line and then perhaps the turtle can be
used to create any drawing then could be made using a pencil. That sets
one path towards difficulty: set out the instructions for a turtle to
approximate some of the pencil work from Leonardo da Vinci or Albricht
Durer! That would probably just ends up as a hugely long list of movements
and although the output would be spectacular the text of the sequence of
instructions to the turtle would not be very interesting or informative!
So here we will concentrate on examples where the sequence of instructions
is reasonably tidy. It was easy to understand what the square and zigzag
scripts would lead to, but the point of this chapter is that with fairly
harmless-looking extensions to the set of operations that a turtle can
be asked to perform it gets remarkably harder to predict what will
emerge or reason about it in detail. So what we provide here are a
selection of more or less difficult questions and challenges ragarding
turtle behaviour and we will not spoil them all by giving all the
answers!

\begin{enumerate}
\item Start with something that is not too hard. For exactly what angles of
turn will a sequence rather like the one that draws a square return to
its starting point, and how many steps will that take? Angles do not need
to be whole numbers of degrees.
\item If the turtle position is computed using computer arithmetic that is
only precise to say around 16 or 17 significant figuures, for a pattern that
would close up in an ideal world how far from joining up can it be in
reality? What are the consequences if the turtle keeps following the same
pattern of activity for a really long time?
\item What rule will lead to the turtle following a nice spiral path,
and how does it behave beyond the time it reaches the centre (of it ever does)?
\item Suppose that at each step the turtle moves forward by unit distance and
then spins so that the next direction is utterly at random. That could include
it keeping on in its original direction, totally backtracking or anything in
between. After $N$ steps about how far from its starting point is it
likely to be?
\item As above, but the new random direction is limited, say to corespond
to turning right by an angle uniformly chosen between 0 and 180 degrees? How
much does this impact things as against the fully random turn?
\item Suppose the turtle has a home and it makes a random turn that is almost
fully random but that has a rather small bias toward pointint it homewards.
How big does this bias be to give it a good change of getting within
a reasonable range of its island? This case can be interpreted as a reasonable
first attempt to model real bird or animal long range migration skills
by exploring just how much navigational precision they actually need. And
the traces of movement of several turtles trying this out can make nice
pictures!
\item For parameters $x$ and $N$ consider the turtle instructions:
\begin{verbatim}
  a = b = c = 0
  repeat N times
    a = a + x
    b = b + a
    c = c + b
    move forward by 1
    turn left by c
\end{verbatim}
This has introduced some arithmetic and is a generalisation of the challenge
to understand what drawing will emerge from ``move 1;turn 1;move 1;turn 2;
move 1; turn 3;\ldots'' where the angle turned at each step grows. Note that
turning by angles over $360^{\circ}$ is perfectly respectable in that you
just spin all the way around once (not havikng any overall effect!) and the
turn by the specified angle less 360.
The challenge here is to understand what values of $x$ lead to closed
paths, how long the paths are before they join up (i.e.\ how large should
$N$ be to make this neat), and what symmetries there will be in the
picture created. For some values of $x$ one gets a 3-fold symmetry. Just
what values of $x$ lead to that? Can one get a 5-fold symmetry ever?
And why are the pictures so decorative?

\end{enumerate}



% This dates from some while ago and is no longer any use as such
% because Java has first deprecated and subsequently removed the "Japplet"
% facility that I used here and that provided a really easy way to set up
% a pane on which "g.drawLine" could do its stuff!
%/*
% *  Turtle.java                                 A C Norman
% *  illustration of Turtle Graphics and the "paint" method.
% */
%
%import javax.swing.*;
%import java.awt.*;
%import static java.lang.Math.*;
%
%public class Turtle extends JApplet
%{
%    public void paint(Graphics g)
%    {   // Try changing the following 3 numbers...
%        double size = 5.0, inc = 11.0;
%        int N = 5000;
%        double x = 200.0, y=200.0,
%               th1 = 0.0, th2 = 0.0, th3  = 0.0;
%        for (int i=0; i<N; i++)
%        {   th3 = th3 + inc;
%            th2 = th2 + th3;
%            th1 = th1 + th2;
%            double x1 = x+size*cos(PI*th1/180.0);
%            double y1 = y+size*sin(PI*th1/180.0);
%            g.drawLine((int)x, (int)y, (int)x1, (int)y1);
%            x = x1;
%            y = y1;
%        }
%    }
%}
%/* end of Turtle.java */


