\chapter{Lazy Programming}
Rather than cut-down ML this is thinking cut-down Miranda or Haskell. And
it introduces a list comprehension notation, as in
\begin{verbatim}
    {x^2 | x in {1,2,...}}
\end{verbatim}
that is used in place of map and that copes with non-terminating lists.
See all the early Miranda-related stuff from Turner for witty examples of
what this can achieve.

\begin{enumerate}
\item Describe the list \verb+{2,3,5,7,11,13,17,...}+ of primes.
\item Interpret a list as the digits in a number expressed in some radix.
So the in decimal the reciprocal of 7 is \verb+{1,4,2,8,5,7,1,...}+.
Do basic arithmetic on the infinite precision representation that this
gives you.
\item Develop code to convert from one radix to another. What is the
reciprocal of seven in octal?
\item Introduce the idea of a mixed radix where the ration between the
weights of columns is not constant. Now \verb+{1,1,1,1,1,...}+ with radix
\verb+{1,2,3,4,...}+ is a way of representing the mathematical constant e, and
if you apply radix conversion you get that in decimal to unlimited
precision.
\end{enumerate}
