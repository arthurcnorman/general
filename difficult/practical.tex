\chapter{Practical Performance}
Galactical algorithms is all about pushing theory and assymptotic
performance estimates to the limit regardless of whether the
result would be practical. So a scheme that ran in $10^{100}N$ when
faced with a problem of size $N$ would be prefereed to one that
needed $0.001 N^{1.0001}$ because for large enough $N$ it would be the
winner. This chapter considers the other end of a spectrum and
considers cases where absolute timings measured in minutes and seconds
or space use in bytes is to be optimized, but in the pursuit of
perfection all other constraints and limitations are ignored. This
can certainly involve use of significantly messy and elaborate code
to achieve even small improvements over straightforward implementation.

\section{Perfect data compression}

\section{Abandoning portability}

\section{Special hardware support}

\section{Composite methods}

\section{Dynamic code generation}

