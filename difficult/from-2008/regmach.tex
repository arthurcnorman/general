\chapter{Register Machines}
\index{Register machines}
\section{Flowcharts}
If you needed to explain about the sorts of step-by-step methods that
are commonly used with computers to solve problems you would quite
probably start by drawing some {\em flowcharts}. These are a traditional
way of explaining how to build up a sequence of tests and actions so that
by threading a path through the chard you trace out steps that solve your
problem. Flowcharts are not just used with computers --- indeed they grew out
of schemes that were to formalise and describe business proceduces in days
before computers had been invented. They commonly appear in pamphlets
explaining how to use or fix problems with all sorts of domestic equipment,
and using then to describe things that everybody already understands quite
well is possible if a bit ridiculous. Traditionally different shaped cells are
written join up using arrows. There ought always to be a cell to start at
and one labelled {\bf stop}.\footnote{Making the pictures for this
article will be something I want to delaY until I have the basic word drafted
--- so I will use CRUDE text-art for now.}

\begin{verbatim}
     (start}
       |   <-----------------------
       /\                 |re-read |
     understood this? --------------
       \/
        | 
     (stop)

\end{verbatim}\footnote{I want some trivial and obvious but slightly jokey
exampe of an irrelevant flowchart here}

\section{Examples}
\begin{enumerate}
\item Multiply a number by a constant
\item Divide by a constant,checking if division is exact
\item Multiply two integers
\item Test equality of two values
\item note consequences of above for packing several
integers into one register and for letting a register machine simulate
actions common in full-scale programming systems
\end{enumerate}

\section{A Harder Problem}
The Busy Beaver\index{Busy Beaver} problem is \ldots

(take home problem is to seek the
second-busiest beaver for a register machine with n states, n = 2,3,\ldots

