\chapter{Lessons that have been learned}


{\em
The comments here are just ACN rambling a bit more

I have just picke dup a copy of Wolfram's New Kind of Science for not
a huge amount on eBay. That is a much bulkier volume than I had expected,
not having checked it before. But what is maybe more terrifing is that
Wolfram now has a ``20 years on'' book that he explains basically as
``When I write ANKoS I thought it was a real breakthrough for the world,
but 20 years on I see it is way better than even the extreme level of
importance I saw in it back them''.

I have only just started reading it and the main think that comes across
is how utterly Wolfram wants to make a point that everybody from before
the ancient Greeks has just skipped past the motherload of overwhelmingly
important stuff that he and he alone has discovered. He is asserting that
what he has discerned upends every scientific and many other disciplines
totally and provides a way to discern the true nature of everything.
Gosh it is amazing in that way. Wow -- what a guy.

However for the purposes of what this book wants to do it is perhaps
rather nice in that I think a major point he is wanting to hammer on is
that something that follows very simple rules can have astonishingly
complicated behaviour, and that this applies not to just one sort
of ``simple thing'' but rather generally.

It is less clear to me (as yet) whether he can them do anything interesting
with the complexity apart from show it off in loads of pictures. And some
of us sort of believed that the not-too-bad equations of fluid dynamics
could lead to very messy turbulence and not just smooth flow, that
looking at multiplication and division dumped one into the quasi
regularity of the distribution of prime numbers and great depth, and
that the investigation simple problems like "boolean satisfiability" could
tell you about all the other NP-complete problems. But that some of these are
discrete and some continuous so he will have to do a really merry dance to
convince me that they are all the same even if all show complicated
behaviours.  But still reviewing all he has to say is liable to reveal
a range of very find examples of things where the starting point is
simple and the end-point really is not.

A lot of what he talks about is cellular automata.

A different thnk to consider is ``solve your problem by first designing and
building your computer, then the software stack\ldots''. This is of course
just what people had to do in the 1940s. And indeed Babbage/Lovelace
had a go there, and there was Konrad Zuse and his relay-based computer
where one can even at least imagine constructing the relays\ldots.

In yet a different direction I think of steam engines. The pistons must slide
nicely into their cylinders so they have to move in straight lines, but if
you support them with sliding supports that migh introduce friction you do not
like. So how do you make a pin-jointed linkage so that the end-point moves
in exactly a straight line? One answer is Hart's Inversor. Now having
invented that if you are a massochist you set up all the simultaneous
equations that characterise the way that the other end of a rod that has
a fixed pivot at one end lies on a circle, and so on. You then see if you
can simplify and solve all those quations to prove that the key endpoint
lies on a straight line. This is a horrid thing to try Groebner Bases on.

} 
 


\bibliographystyle{plain}
\bibliography{full}
\input full.ind

