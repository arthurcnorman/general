\chapter{Logic Programming}
The threshold concept here is that of separating concern about ``what''
from worry about ``how''. Well if you view it as a mathematical notation
(ie Horn Clauses) Prolog has no syntax to mention and its meaning is
merely (ha ha) logic, so it is trivial. I could cobble up an ultimately
cut-down pico-prolog in pretty short order for use here. The other key
words here are {\em logic}, {\em consequence}, {\em satisfy} and {\em
back-track}.

\begin{enumerate}
\item Typical Sunday newspaper puzzles along the lines of {\em who owns the
zebra?}
\item The use of a predicate ``in reverse'', so that eg {\em add} can also
be used as {\em find all pairs of integers that sum to this value}.
\item The interpretation of clauses as rows in a database and computations as
queries, applied (say) to a family-tree database or some other structured
data where chaining is interesting.
\item Some sort of bin-packing or jigsaw-style puzzle
\item 8 queens
\end{enumerate}

