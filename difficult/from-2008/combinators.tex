\chapter{Combinators}
The simple (if not instantly obvious) rewrite system of combinators has
long featured in our Foundations of Functional Programming course, and
the fact that just two symbols (S and K) with simple rules for how they
behave gives you a computationally universal system is so very neat. Some
of us also like the fact that the polymorphic types one gives to S and K
correspond to the axioms of propositional calculus, so well-typed
combinations and theorems in logic are directly related! But that maybe
is not GCSE-level!


\begin{enumerate}
\item I like the exam question that asks e.g. what S S S S S ... (for n copies of S) would reduce to. If might even be accessible in a single-day session!
\item What is S K K x
\item if I = S K K, what is S I I x
\item what about (S I I) (S I I)
\item we are by now amazingly close to being able to define the fix-point operator Y
\item Could one get to Church numerals?
\end{enumerate}

