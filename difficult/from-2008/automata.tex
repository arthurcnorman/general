\chapter{Finite Automata}
This is a topic where the problem solving part may fit happily as a
pencil and paper exercise rather than needing computers. The lecture part
can introduce state machines. Key ideas about them (and which of these
can fit into a presentation to novices is not quite clear yet) are the
relationship between a machine and a language, non-determinism and ways
of composing automate to make larger or different ones. The word
{\em Chomsky} can be mentioned maybe. One perhaps attractive way to go is
then to look at cellular automata --- ie rows or grids of finite state
machines.

\begin{enumerate}
\item Wolfram's ``New kind of science'' can be raided for examples
\item Miller's Periodic Forests of Stunted Trees may be fun to draw, but
their relationship to the factors of polynomials over GF(2) may be beyond
GCSE level!
\item The {\em firing squad} problem might be a good take-home one?
And providing a download-from-web applet to animate things and help
visualise may be good
\item One can invent automate design puzzles easily by specifying a
regular language (eg via either a regular expression or a
non-deterministic automata) and asking people to design a DFA for it.
Asking for the smallest DFA raises the bar further.
\end{enumerate}

